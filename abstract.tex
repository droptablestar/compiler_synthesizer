\NeedsTeXFormat{LaTeX2e}
\documentclass[11pt]{article}
\usepackage{times,url,amsmath,mathpazo,xltxtra,fancyhdr,xunicode,hyperref,
  amssymb,amsthm,mathpartir,exercise,fontspec,newunicodechar,alltt}

\usepackage[nohead,nofoot,vmargin=1.25in,hmargin=1.25in]{geometry}

\input{/Users/jreese/Documents/uni/commands.tex}
\date{}

\title{Synthesising A Compiler\author{Josh Reese}}

\begin{document}
\maketitle
Imagine designing and implementing a programming language but not
having to write a compiler or interpreter for it. In this talk we will
present a system which can be used for the automatic generation of a
compiler. The idea for this work comes from the area of program
synthesis which aims to take a partially completed program, or a
program with `holes', and complete the program.

In order to demonstrate this we have designed two simple
languages. One which contains various bit of syntactic `sugar' and one
which does not. From the specifications of these two languages, and
the use of various simple input programs, we are able to automatically
generate a compiler from the higher level language to the lower level
language.  
\end{document}
